\documentclass{article}

\usepackage{amsmath}
\usepackage{enumitem}
\usepackage{tikz-timing}
\usepackage{graphicx}

\graphicspath{{./images/}}

\begin{document}
	
	
	2.2. Explain what are the differences between latches and flip flops with your own words. \\
	The main difference between a latch and a flip flop is that the first is asynchronous, and the second is synchronized with a clock. When, for instance, a set signal is given to latch, the time required to switch states will only depend on the internal delays of gate. Flip flops, however, are triggered either by positive or negative edge of clock signal, according to their type. Implementation of synchronization requires more logic gates. \\
		
 	2.4. Construct the truth table of an SR latch which does not have an Enable input. 
 	
 	\begin{center} 
 	\begin{tabular}{|c|c|c|c|}
	 	\hline
	 	\rule{0pt}{1\normalbaselineskip}S & R & $Q(t+1)$ & $\overline{Q}(t+1)$ \\
		\hline	
		\rule{0pt}{1\normalbaselineskip}0 & 0 & $Q(t)$ & $\overline{Q}(t)$ \\
		1 & 0 & 1 & 0 \\
		0 & 1 & 0 & 1 \\
		1 & 1 & $x$ & $x$ \\
		\hline
	\end{tabular}
	\begin{tabular}{r}
	\rule{0pt}{1\normalbaselineskip}\\
	\rule{0pt}{1\normalbaselineskip}(preserve)\\
	(set)\\
	(reset)\\
		(forbidden) \\
	\end{tabular}
	\end{center}
	where $Q(t)$ and $Q(t+1)$, $\overline{Q}(t)$ and $\overline{Q}(t+1)$ represent previous and next states, and $x$ is unknown. \\
	
	2.6. Construct the truth table of a D flip flop.	\\
	\begin{center} 
 	\begin{tabular}{|c|c|c|c|}
	 	\hline
		\rule{0pt}{1\normalbaselineskip}D & CLK & $Q(t+1)$ &  $\overline{Q}(t+1)$ \\
		\hline	
		\rule{0pt}{1\normalbaselineskip}$\phi$ & $\phi$ & $Q(t)$ & $\overline{Q}(t)$ \\
		1 & \texttiming{LH} & 1 & 0 \\
		0 & \texttiming{LH} & 0 & 1 \\
		\hline
	\end{tabular}
	\begin{tabular}{r}
	\rule{0pt}{1\normalbaselineskip}\\
	\rule{0pt}{1\normalbaselineskip}(preserve)\\
	(set)\\
	(reset)\\
	\end{tabular}
	\end{center}
	
	where $Q(t)$ and $Q(t+1)$, $\overline{Q}(t)$ and $\overline{Q}(t+1)$ represent previous and next states, $\phi$ is any state and \texttiming{LH} symbol means rising edge, i.e. transition from low to high. \\
	
	\textbf{Part 1:}
	\begin{figure}[h]
		\centering
		\includegraphics[width=0.8\textwidth]{part1.png}
		\caption{SR Latch simulation}
		\centering
	\end{figure}	
	
	The behaviour of SR Latch can be expressed with an equation of form $Q(t + 1) = f(S; R; Q(t)).$ From the truth table, all combinations of S and R are selected except the forbidden state, and $Q(t + 1)$ is expressed and simplified using rules of Boolean algebra as follows:

	\begin{align*}			 
	 Q(t + 1) &= \bar{S} \bar{R} Q(t) + \bar{S} R \cdot 0 + S \bar {R} \cdot 1 \tag{Truth table} \\
	 &= \bar{S} \bar{R} Q(t) + 0 + S  \bar{R} \cdot 1 \tag{Dominance} \\
	 &= \bar{S} \bar{R} Q(t) + S  \bar{R} \tag{Identity} \\
	 &= \bar{S} \bar{R} Q(t) + S \bar{R} + \bar{R} \bar{R} Q(t) \tag{Consensus theorem} \\
	 &= \bar{S} \bar{R} Q(t) + S \bar{R} + \bar{R} Q(t) \tag{Idempotency} \\
	 &=  Q(t)(\bar{S} \bar{R} + \bar{R}) + S \bar{R}  \tag{Distributivity} \\
	 &=  Q(t)\bar{R} + S \bar{R}  \tag{Absorption} \\
	 &=  Q(t)\bar{R} + S\\
	\end{align*}

	In the final step, $\bar{R}$ is omitted because state $S = 1, R = 1$ is forbidden and should never happen. Therefore omitting it does not change the truth table. To summarize, SR Latch's output is $1$ whenever S is on and R is off, is $0$ whenever R is on and S is off, and it maintains the previous state if both S and R go low. It is therefore called a memory circuit. \\
	
	\textbf{Part 3:}
	\begin{figure}[h]
		\centering
		\includegraphics[width=0.8\textwidth]{part3.png}
		\caption{Negative edge triggered D flip-flop simulation}
		\centering
	\end{figure}	
	
	As it is seen from the wave form, D flip-flop can change state only on the falling clock edge.

	\textbf{Part 6:}
	\begin{figure}[h]
		\centering
		\includegraphics[width=1\textwidth]{part6.png}
		\caption{Synchronous up counter simulation}
		\centering
	\end{figure}	
	
	The truth table is as follows:\\

	\begin{tabular}{|c|c|c|c|}
	\hline
	\rule{0pt}{1\normalbaselineskip} CLK & Reset & $\{A_3, A_2, A_1, A_0\}$ &  $\{A_3, A_2, A_1, A_0\}^+$ \\\hline
	$\phi$ & \texttiming{LH} & $\{\phi,\ \ \phi,\ \  \phi,\ \  \phi\}$ & $\{0,\ \ 0,\ \ 0,\ \  0\}$\\ 
	\texttiming{LH} & 0 & $\{0,\ \ 0,\ \ 0,\ \  0\}$ & $\{0,\ \ 0,\ \ 0,\ \  1\}$\\ 
	\texttiming{LH} & 0 & $\{0,\ \ 0,\ \ 0,\ \  1\}$ & $\{0,\ \ 0,\ \ 1,\ \  0\}$\\ 
	\texttiming{LH} & 0 & $\{0,\ \ 0,\ \ 1,\ \  0\}$ & $\{0,\ \ 0,\ \ 1,\ \  1\}$\\ 
	\texttiming{LH} & 0 & $\{0,\ \ 0,\ \ 1,\ \  1\}$ & $\{0,\ \ 1,\ \ 0,\ \  0\}$\\ 
	\texttiming{LH} & 0 & $\{0,\ \ 1,\ \ 0,\ \  0\}$ & $\{0,\ \ 1,\ \ 0,\ \  1\}$\\ 
	\texttiming{LH} & 0 & $\{0,\ \ 1,\ \ 0,\ \  1\}$ & $\{0,\ \ 1,\ \ 1,\ \  0\}$\\ 
	\texttiming{LH} & 0 & $\{0,\ \ 1,\ \ 1,\ \  0\}$ & $\{0,\ \ 1,\ \ 1,\ \  1\}$\\ 
	\texttiming{LH} & 0 & $\{0,\ \ 1,\ \ 1,\ \  1\}$ & $\{1,\ \ 0,\ \ 0,\ \  0\}$\\ 
	\texttiming{LH} & 0 & $\{1,\ \ 0,\ \ 0,\ \  0\}$ & $\{1,\ \ 0,\ \ 0,\ \  1\}$\\ 
	\texttiming{LH} & 0 & $\{1,\ \ 0,\ \ 0,\ \  1\}$ & $\{1,\ \ 0,\ \ 1,\ \  0\}$\\ 
	\texttiming{LH} & 0 & $\{1,\ \ 0,\ \ 1,\ \  0\}$ & $\{1,\ \ 0,\ \ 1,\ \  1\}$ \\ 
	\texttiming{LH} & 0 & $\{1,\ \ 0,\ \ 1,\ \  1\}$ & $\{1,\ \ 1,\ \ 0,\ \  0\}$  \\ 
	\texttiming{LH} & 0 & $\{1,\ \ 1,\ \ 0,\ \  0\}$ & $\{1,\ \ 1,\ \ 0,\ \  1\}$  \\ 
	\texttiming{LH} & 0 & $\{1,\ \ 1,\ \ 0,\ \  1\}$ & $\{1,\ \ 1,\ \ 1,\ \  0\}$  \\ 
	\texttiming{LH} & 0 & $\{1,\ \ 1,\ \ 1,\ \  0\}$ & $\{0,\ \ 0,\ \ 0,\ \  0\}$  \\ 
	
	\hline
\end{tabular}		
	

\end{document}